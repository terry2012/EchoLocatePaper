% ABSTRACT

% Write abstract here.
% Header information is handled by \LaTeX{} magic.

\section{Introduction}
\label{sec:intro}
Operators are currently in the process of evaluating or commercially deploying Voice over LTE (VoLTE) services, which promise to bring unprecedented quality of experience (QoE) to customers and the number of VoLTE deployments has reached 40 worldwide by the end of 2015~\cite{volte_number}. The importance of voice call is evident from almost all major US carriers and social networking platforms today offering some form of Internet calling capability (e.g., Skype, AT\&T's HD Voice, Hangouts).

The key difference between the VoLTE service offered by operators and the \textit{over the top} (OTT) VoIP services are that VoLTE relies on specialized support in both end devices and network architecture, as well as dedicated radio resources during the voice call to achieve the goal of delivering better QoE than the OTT VoIP services that run on the legacy packet-switched network infrastructure. However, previous work~\cite{jia2015performance} has identified various types of problems in the commercially deployed VoLTE services, causing QoE problems such as high unintended call drop rate and long audio muting, which 
fails to fulfill the promise of better QoE for VoLTE call. Yet, despite the growing importance and adoption of VoLTE call, there have been few systematic studies \textit{at scale} that analyze (1) what are the typical QoE issues in the real world, and (2) what are the root causes behind them, due to a lack of practical QoE diagnosis support. 

Diagnosing QoE problems is challenging in two aspects: (1) how to capture QoE problems precisely in the real world, (2) how to get down to the root cause of the captured QoE problems. Metrics such as call drop rate, call setup time have been used by previous work~\cite{jia2015performance} to model the QoE of a voice call, however,  critical user experience such as audio quality is missing. Recent work~\cite{jiang2016improving} studying the call quality of Skype models the QoE using the user's feedback that reflects the quality on a discrete 5-point scale. However, we argue that such immediate user feedback may not be available for all the voice call services, and even in Skype, it's only available for a small fraction of calls, which may be biased. Thus, a set of QoE metrics that are generally applicable and reasonably accurate are required to better capture QoE problems in the real world.

The challenges of root cause diagnosis of VoLTE call compared with the OTT VoIP services are: (1) more complex architecture: multiple entities are involved; (2) more complex lifecycle of the voice call: various techniques are adopted to support, making it a complex service (CSFB, SRVCC, handover, cross-carrier support, etc.) A practical diagnosis system has to bridge the disconnected entities and layers to locate root causes that may reside either in individual layers or in the cross-layer and cross-entity interactions. However, current diagnosis systems~\cite{jiang2016improving} mainly performs offline analysis on the network traces collected from the backend, and doesn't have enough visibility into the issues in the device and access network, not to mention problems occur in the lower network layers on the handset, which makes the diagnosis not only insufficient but also inaccurate. 

As a first step to address these challenges, we propose \name, which is a system that contains both on-device performance indicator capturing client and backend QoE problem diagnosis support. The on-device client integrates support from OEMs that reports critical lower layer performance indicators from both control (SIP, RRC) and data (RTP) plane, which provides us with unprecedented visibility into the life cycle of a voice call. The collected information regarding each voice call from end user is reported to the backend, where machine learning approach is adopted to \texttt{automatically} (1) extract the characteristics of various QoE problems and build new QoE metrics; (2) learn the critical features of each QoE metrics and diagnosis the root cause the corresponding QoE problems; (3) provide actionable suggestions for operators, protocol designers and also device manufacturers to solve the problems. 







